На сегодняшний день существует большое количество фреймворков, предназначенных для построения систем автоматической обработки текста на естественном языке. В состав некоторых фреймворков входят модули графематического анализа. Существуют различные методы, применяемые при выполнении графематического анализа: использование правил, словарный поиск, машинное обучение, эвристика.
\begin{table}[H] \small
	\centering
	\label{t:thyp_gd1}
	\begin{tabular}{ | c | c | c | c |}
		\hline
		Название 							& Метод 				& Языки 		& Платформа 				\\ \hline
		АОТ\cite{web.aot}					& словарный поиск		& русский,		& GNU/Linux,			\\
											&						& английский	& Microsoft Windows(C++)\\ \hline
		Stanford CoreNLP\cite{web.corenlp}	& эвристика				& английский	& JVM (Java)			\\ \hline
		Apache OpenNLP\cite{web.opennlp}	& использование правил,	& английский	& JVM (Java)			\\
											& машинное обучение		& 				&						\\ \hline
		FreeLing\cite{web.freeling}			& использование правил	& русский,		& GNU/Linux (C++)		\\
											&						& английский,	&						\\
											&						& и др.			&						\\ \hline
		Solarix\cite{web.solarix}			& использование правил	& русский,		& GNU/Linux,			\\ 
											&						& английский	& Microsoft Windows(C++)\\
		\hline
	\end{tabular}
	\caption{Проекты, включающие модули графематического анализа}
\end{table}
Существует подход, основанный на использовании правил, задающих границы словоформ. Модули, основанные на использовании правил, обычно требуют для своей работы библиотеку правил. Такие библиотеки могут составляться отдельно для разных языков. Обычно, правила представляют собой регулярные выражения, задающие границы словоформ. Словарный поиск заключается в поиске словоформ в морфологическом словаре. Машинное обучение широко применяется в системах семантического анализа как на этапе разметки текста, так и на последующих фазах. Использование данного подхода сопряжено с составлением обучающего множества. Обычно выполняется совмещение нескольких подходов, для чего используются некоторые эвристические правила.

\subsection{Необходимость нового подхода}
Среди существующих решений нет решений, удовлетворяющих одновременно всем поставленным требованиям:
\begin{enumerate}
	\item 
	возможность разметки текста, содержащего дефекты;
	\item
	возможность одновременной работы с несколькими языками;
	\item
	представление результата в виде ссылок на записи морфологического словаря.
\end{enumerate}
Существующие модули, выполняющие графематический анализ, разрабатываются в составе фреймворков, включающих также модули, выполняющие другие фазы автоматической обработки текста. В связи с этим, на данные модули наложены некоторые ограничения, обусловленные глобальным подходом к обработке текста на естественном языке, применяемом в том или ином фреймворке. В разрабатываемой системе семантического анализа используется подход к анализу естественного языка, лишь частично согласующийся с существующими подходами. Ввиду этого, разрабатываемый модуль предсинтаксического аннотирования текста должен решать дополнительные задачи, непредусмотренные в существующих модулях. Примером такой задачи можно назвать связывание найденных токенов с узлами графового словаря, который содержит узлы разных типов и позволяет производить связывание словоформ со смысловыми узлами. Также следует отметить, что существующие модули графематического анализа работают лишь с некоторыми языками, вследствие чего может понадобиться реализация модуля определения языка, что усложняет архитектуру системы. Ввиду сказанного необходимо разработать новый подход и основанный на нем программный модуль.