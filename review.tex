На сегодняшний день существует большое количество фреймворков, предназначенных для построения систем автоматической обработки текста на естественном языке. В состав некоторых фреймворков входят модули графематического анализа. Существуют различные методы, применяемые при выполнении графематического анализа:

\begin{itemize}
\item
использование правил;
\item
словарный поиск;
\item
машинное обучение;
\item
эвристика.
\end{itemize}

\begin{table}[H] \small
	\centering
	\label{t:thyp_gd1}
	\begin{tabular}{ | c | c | c | c |}
		\hline
		Название 							& Метод 				& Языки 		& Платформа 				\\ \hline
		АОТ\cite{web.aot}					& словарный поиск		& русский,		& GNU/Linux,			\\
											&						& английский	& Microsoft Windows(C++)\\ \hline
		Stanford CoreNLP\cite{web.corenlp}	& эвристика				& английский	& JVM (Java)			\\ \hline
		Apache OpenNLP\cite{web.opennlp}	& использование правил,	& английский	& JVM (Java)			\\
											& машинное обучение		& 				&						\\ \hline
		FreeLing\cite{web.freeling}			& использование правил	& русский,		& GNU/Linux (C++)		\\
											&						& английский,	&						\\
											&						& и др.			&						\\ \hline
		Solarix\cite{web.solarix}			& использование правил	& русский,		& GNU/Linux,			\\ 
											&						& английский	& Microsoft Windows(C++)\\
		\hline
	\end{tabular}
	\caption{Проекты, включающие модули графематического анализа}
\end{table}

\subsection{Использование правил}
Данный подход основан на использовании правил, задающих границы словоформ. Модуль, основанный на использовании правил, обычно требует для своей работы библиотеку правил. Такие библиотеки могут составляться отдельно для разных языков. Обычно, правила представляют собой регулярные выражения, задающие границы словоформ.
\subsubsection{Регулярные выражения}
\subsection{Словарный поиск}
\subsection{Машинное обучение}
\subsection{Эвристика}

\subsection{Необходимость нового подхода}
Существующие модули, выполняющие графематический анализ, разрабатываются в составе фреймворков, включающих модули, выполняющие другие фазы автоматической обработки текста. В связи с этим, на данные модули наложены некоторые ограничения, обусловленные глобальным подходом к обработке текста на естественном языке, применяемом в том или ином фреймворке. В разрабатываемой системе семантического анализа используется подход к анализу естественного языка, лишь частично согласующийся с существующими подходами. Ввиду этого, разрабатываемый модуль предсинтаксического аннотирования текста должен решать дополнительные задачи, непредусмотренные в существующих модулях. Примером такой задачи можно назвать связывание найденных токенов с узлами графового словаря, содержащего узлы разных типов и позволяющий производить связывание словоформ со смысловыми узлами. Так же следует отметить, что существующие модули графематического анализа работают лишь с некоторыми языками, вследствие чего может понадобиться реализация модуля определения языка, что усложняет архитектуру системы.