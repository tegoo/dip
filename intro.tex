Среди специалистов в области психолингвистики, когнитивной психологии и философии познания существует гипотеза о существовании  «языка мысли» или семантического языка. Этот язык является представлением мысли, а перевод с естественного языка на семантический является процессом понимания. Семантический язык универсален и присущ всем людям с момента рождения. В своей книге «Язык мысли» («The Language of Thought») Джерри Фодор называет этот язык «Ментализом». Неотъемлемой частью процесса восприятия естественного языка является распознавание слов. Человек способен распознать слова в написанном тексте, даже при отсутствии разделительных символов и наличии орфографических ошибок. Специалисты в области когнитивной психологии достаточно давно изучают психологические аспекты распознавания слов в тексте и формулируют различные модели этого процесса: распознавание по форме, последовательное и параллельное буквенное распознавание. 
Одной из задач области обработки текста на естественном языке является получение некоторого машинного представления смысла обрабатываемого текста. Решение данной задачи включает в себя несколько фаз обработки текста, на первой из которых необходимо произвести распознавание слов. 