\paragraph{}
Среди специалистов в области психолингвистики, когнитивной психологии и философии познания существует гипотеза о существовании  «языка мысли» или семантического языка. Этот язык является представлением мысли, а перевод с естественного языка на семантический является процессом понимания. Семантический язык универсален и присущ всем людям с момента рождения. В своей книге «Язык мысли» («The Language of Thought») Джерри Фодор называет этот язык «Ментализом». Неотъемлемой частью процесса восприятия естественного языка является распознавание слов. Человек способен распознать слова в написанном тексте, даже при отсутствии разделительных символов и наличии орфографических ошибок. Специалисты в области когнитивной психологии достаточно давно изучают психологические аспекты распознавания слов в тексте и формулируют различные модели этого процесса: распознавание по форме, последовательное и параллельное буквенное распознавание. 
\paragraph{}
Одной из задач области обработки текста на естественном языке является получение некоторого машинного представления смысла обрабатываемого текста. Решение данной задачи включает в себя несколько фаз обработки текста, на первой из которых необходимо произвести распознавание слов.

\newpage
\subsection{Описание предметной области}
Обработка текста на естественнном языке или компьютерная лингвистика - междисциплинарное научное направление объединяющее знания и методы лингвистики и области искусственного интеллекта. Объектом исследования компьютерной лингвистики является текст на естественном языке. Текст можно рассматривать как выражение смысла или мысли. Следует отметить, что для одной и той же мысли может существовать несколько формулировок на естественном языке. Этот факт является аргументом в пользу справедливости гипотезы о существовании “языка мысли”. Постижение смысла, содержащегося в обрабатываемом тектсе, и формирование его некоторого формального машинного представления и есть одна из основных задач компьютерной лингвистики. Возможность получения формального представления смысла является необходимой предпосылкой для создания ителлектуальных информационных систем. В рамках компьютерной лигвистики существуют различные математические модели, используемые для описания естественного языка. Что-нибудь про Хомского. В современной компьютерной лингвистике большое применение имеют вероятностные модели языка.

\subsection{Цель работы}

\subsection{Актуальность решаемой задачи}