\paragraph{}
Среди специалистов в области психолингвистики, когнитивной психологии и философии познания существует гипотеза о существовании  «языка мысли» или семантического языка. Этот язык является представлением мысли, а перевод с естественного языка на семантический является процессом понимания. Семантический язык универсален и присущ всем людям с момента рождения. В своей книге «Язык мысли» («The Language of Thought»\cite{fodor}) Джерри Фодор называет этот язык «Ментализом». Неотъемлемой частью процесса восприятия естественного языка является распознавание слов. Человек способен распознать слова в написанном тексте, даже при отсутствии разделительных символов и наличии орфографических ошибок. Специалисты в области когнитивной психологии достаточно давно изучают психологические аспекты распознавания слов в тексте и формулируют различные модели этого процесса: распознавание по форме, последовательное и параллельное буквенное распознавание\cite{larson}. 
\paragraph{}
Одной из задач области обработки текста на естественном языке является получение некоторого машинного представления смысла обрабатываемого текста. Решение данной задачи включает в себя несколько фаз обработки текста, на первой из которых необходимо произвести распознавание слов.

\newpage
\subsection{Описание предметной области}
Обработка текста на естественнном языке или компьютерная лингвистика - междисциплинарное научное направление, объединяющее знания и методы лингвистики и области искусственного интеллекта. Объектом исследования компьютерной лингвистики является текст на естественном языке. Текст можно рассматривать как выражение смысла или мысли. Следует отметить, что для одной и той же мысли может существовать несколько формулировок на естественном языке. Этот факт является аргументом в пользу справедливости гипотезы о существовании “языка мысли”. Постижение смысла, содержащегося в обрабатываемом тектсе, и формирование его некоторого формального машинного представления и есть одна из основных задач компьютерной лингвистики. Возможность получения формального представления смысла является необходимой предпосылкой для создания ителлектуальных информационных систем. В рамках компьютерной лигвистики существуют различные математические модели, используемые для описания естественного языка. В современной компьютерной лингвистике большое применение имеют вероятностные модели языка.

\paragraph{}
В рамках автоматической обработки текста выделяют следующие фазы:
\begin{itemize}
\item
Графематический анализ;
\item
Морфологический анализ;
\item
Синтаксический анализ;
\item
Семантический анализ.
\end{itemize}


\subsection{Цель работы}
Разработать подход предсентактического аннотирования текста, позволяющий выполнять раземтку текста. Разработанный подход должен позволять выполнять аннотирование текста, содержащего дефекты. На основе найденного подхода необходимо реализовать программный модуль предсинтаксического аннотирования текста. Программный модуль должен интегрироваться в разрабатываемую систему семантического анализа. Подход должен позволять адаптировать программный модуль для выполнения аннотирования текстов на разных алфавитных языках.

\subsection{Актуальность решаемой задачи}
Поставленная задача актуальна ввиду специфичных требований к модулю предсинтаксического аннотирования текста, связанных с подходом разрабатываемым в рамках системы семантического анализа. 
К специфичным требованиям относятся следующие требования
устойчивость к дефектам 
возможность адаптирования для работы с новыми алфавитными языками
результат работы модуля должен ссылаться на словарные статьи графового словаря, используемого в системе семантического анализа
Устойчивость к дефектам позволит системе семантического анализа обрабатывать тексты, не проходящие предварительную орфографическую проверку. Такие текты наиболее широко распространены в различных веб-ресурсах, веб-форумах, где пользователи набирают текс в спешке и не уделяют должного внимания проверке своих текстов. Анализ таких текстов представляет широкий интерес для различных информационных систем и может быть использован, например,  для исследования паттернов поведения пользователей. Другим примером текстов, содержащих дефекты является тексты, полученные в результате оптического сканирования. К данному классу текстов относятся тексты, существующие на бумажном носителе и не доступные в электронном виде, например, архивные документы. Анализ таких текстов так же представляет штрокий интерес для информационных систем, используемых в некоторых сферах.