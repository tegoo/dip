\documentclass[14pt]{extarticle}

\usepackage[T2A]{fontenc}               % fonts
\usepackage[utf8x]{inputenc}             % UTF-8
\usepackage[english,russian]{babel}     % russian
\usepackage{cmap}                       % russian search in pdf

\usepackage[a4paper, left=3cm, right=2cm, top=2cm, bottom=3cm]{geometry}   % pagesize


\begin{document}

\section{Титул}
Здравствуйте, уважаемая аттистационная комиссия. 
Представляю вам мою дипломную работу.


\section{Задача предсинтаксического аннотироания тектса}


\section{Существующие подходы}


\section{Цель}


\section{Разработанный подход}
В данной работе предлагается подход к предсинтаксическому аннотированию текста на естественном языке. Предлагаемый подход состоит из двух фаз: фазы разметки текста и фазы словарного поиска. Первая фаза введена для выполнения предварительной разметки текста. Учет и коррекция дефектов в словоформах достигается введением второй фазы разметки.

\section{Фаза разметки текста}
На фазе разметки текста решается задача графематического анализа текста, представляющего собой последовтаельность символов пунктуации и символов алфавита естественного языка. Результатом является последовательность найденных токенов.

\section{Скрытая Марковская Модель (СММ)}


\section{Фаза словарного поиска}


\section{Графовый словарь}
В разработанном подходе используется словарь, имеющий графовую структуру.  Данный словарь объединяет в себе морфологический словарь и семантическую сеть. В рамках графовой структуры словаря определены узлы трех типов: семантические узлы, узлы лексем и узлы словоформ. Узлы словаря связаны ребрами следующих типов: ребра, соответствующие семантическим отношениям


\section{Алгоритм нечеткого поиска}


\section{Практическая реализация}


\section{Обучающее множество}


\section{Результаты исследования}

\end{document}