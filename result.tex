В результате исследования был найден подход к предсинтаксическому аннотированию текста, соответствующий поставленным требованиям:
\begin{enumerate}
	\item 
	возможность разметки текста, содержащего дефекты;
	\item
	возможность одновременной работы с несколькими языками;
	\item
	представление результата в виде ссылок на записи морфологического словаря.
\end{enumerate}

Предложен двухэтапный алгоритм, построенный на основе использования скрытой марковской модели и нечеткого поиска по морфологическому словарю.

Разработан программный модуль, выполняющий предсинтаксическое аннотирование текста на естественном языке. В ходе тестирования определены требования к объему обучающего множества, используемого для обучения скрытой марковской модели.